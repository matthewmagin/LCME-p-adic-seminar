\documentclass[a4, 12pt]{article}
\usepackage{xcolor}
\usepackage{pagecolor}
\usepackage{lipsum}
\setlength\parindent{0pt}
\pagecolor{black!80}
\color{white}
\renewcommand{\bf}{\textbf}
\usepackage{enumitem}
\newcommand{\Mod}[1]{\ (\mathrm{mod}\ #1)}
\usepackage{amsmath,amsfonts,amssymb,amsthm,mathtools}

\usepackage[left=0.5cm,right=0.5cm,
    top=1.5cm,bottom=1.5cm,bindingoffset=0.5cm]{geometry}
\usepackage[english,russian]{babel}
\begin{document}

\begin{center}
\section*{Научный семинар по теории чисел, ЛНМО, 2022/2023.}
\subsection*{Задачи и упражнения.}
\end{center}
\begin{center}
    \textsc{Элементарная теория чисел.}
\end{center}

\bf{1.} Докажите, что числа Ферма $F_n = 2^{2^n} + 1$ попарно взаимно просты. \\
\emph{Указание.} Удобно доказать и воспользоваться рекуррентной формулой для чисел Ферма. \\

\bf{2.} Для натурального $m > 1$ вычислите в кольце $\mathbb{Z}/m\mathbb{Z}$:
\begin{itemize}
    \item Сумму всех элементов.
    \item Сумму квадратов всех элементов.
    \item Сумму всех попарных произведений  элементов.
    \item Сумму всех обратимых элементов.
    \item Сумму квадратов всех обратимых элементов.
\end{itemize}

\bf{3.} Докажите \emph{теорему Вильсона.} Сравнение
\[ (p - 1)! + 1 \equiv 0 \Mod{p} \]
выполняется тогда и только тогда, когда $p$~---  простое. \\

\bf{4.} Для каких простых чисел $p$ разрешимо сравнение $x^2 + 3 \equiv 0 \Mod{p}$?\\

\bf{5.} Докажите, что все решения сравнения $x^2 + 1 \equiv 0 \Mod{p}$, где $p = 4k + 1, k \in \mathbb{N}$ имеют вид
\[ x = \pm 1 \cdot 2 \cdot \ldots \cdot 2m \Mod{p} \]
\emph{Указание.} Удобно воспользоваться теоремой Вильсона



\end{document}

